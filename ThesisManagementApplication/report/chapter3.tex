\chapter{Requirement Specifications} \label{chap:reqs}

%\version{v1.10.2015}

\section*{}
\section{Existing System}
	The existing system is entirely based on paper work,which may vulnerable to human error.
	\par In existing system students register themselves manually for thesis.Than they need to find out the supervisor for thesis.In existing system you have to communicate to supervisors by checking supervisor is available or not.With a manual operating system students rely on regular contact with their supervisor which is basically very time consuming.Submitting proposal/Thesis manually which have a lot of paper work involved and also their is no proper place to save the students records. 
The main problem areas of existing systems are following
\begin{itemize}






       \item Manually registration
       \item Manually proposal and thesis submission
       \item A lot of paper work
       \item No security
       \item Time consuming
       \item No proper place to store student record
       \item Human error chances 
       \item Very slow to operate 
      \end{itemize}
       
	
	\section{Proposed System}
	The main purpose of this project is convert current Manuel thesis system to automated thesis system, which involved no paper work.This can be done by making a user-friendly interface for admin and student so that both users can perform their task quickly and efficiently. 
	In this project we use these following technologies:
	\begin{itemize}
		\item C\#
		\item XML
		\item SQL Database 
	\item JavaScript HTML CSS
	\end{itemize}
	with following key feature:
	
	\begin{itemize}
		\item Accurate result
	\item Faster and more accessible
	\item User friendly
	\item e-folder facility
	\end{itemize}
	
This Thesis system is basically design to facilitate faculty members and students, so that students can easily contact with their supervisor through this system.In this system students can register themselves and get a link of student portal where they can select his supervisor of his choice.After student suggestion system's administration side will decide whether the supervisor is available or not, if YES than assign that supervisor otherwise assign by their choice.Then thesis will be submitted by students to relevant supervisors. Internal and external examiners will be appointed by administration and their data will be updated on this system. Administrator will evaluate thesis and will update results on this portal.In this system e-folder facility is there all the record of students and thesis examiner data save in this e-folder, so this also insure some kind of your data security. 
\par The main objective of this application is able to facilitate the students and Faculty members  of the University. This application will allow the Faculty members to save students recodes in database in proper way and get rid off from paper work system.
	
	\section{Requirement Specifications}
	
	\begin{enumerate}
		\item User Requirements: The user of the system which can have a contact with system directly or indirectly are identified below:
	\par Developer: Complete control of all aspects of his application. Any time he makes changes in application like update or change visual interaction.
	\par Technical User: The user who have full command on databases and uses of other programmed applications. This is naive user of application.
	\par Non-Technical User: These are end users of the application who only read specific file according to there need.
	\enlargethispage{-\baselineskip}
	\item System Requirements: 
	\begin{itemize}
			\item Functional Requirements:
			\begin{enumerate}
		\item Save large numbers of files on server
	\item Database can only access by administrator
	\item Maintaining the detail of examiner and student in the database
	\item The system will allow access to students to submit proposal/thesis for a specific duration of time
	\end{enumerate}
	\item Non-functional Requirement:
	\begin{enumerate}
		\item  Usability
	\item Performance
	\item Capacity.
	\item Maintenance.
	\item Operations.
	\item Security.
	\item Attractive layout
	\end{enumerate}
		\end{itemize}
	
		\item Software Requirements:
	The software requirements for this application is shown in Table \ref{Software Requirements}.
	\begin{table}[]
\centering
\begin{tabular}{|l|l|l|}
\hline
 \bfseries Development Tool & \bfseries Operating System & \bfseries Database System \\ 
\hline
Visual Studio 2015 & Windows XP, 7, 8, 8.1, 10 & Microsoft SQL Server  \\
\hline
Microsoft Sql Server Management Studio & &  \\
\hline
\end{tabular}
\caption{Software Requirements}
\label{Software Requirements}
\end{table}

\item Hardware Requirements: The hardware requirements for this application is shown in Table \ref{Hardware Requirements}.
\begin{table}[]
\centering
\begin{tabular}{|l|l|}
\hline
\bfseries Hardware          & Client                 \\
\hline
\bfseries Processor         & Dual Core 1.8 or above \\
\hline
\bfseries Memory            & 1GB or above           \\
\hline
\bfseries Hardware Capacity & 40GB or above          \\
\hline
\bfseries Desktop or Laptop & YES \\
\hline                   
\end{tabular}
\caption{Hardware Requirements}
\label{Hardware Requirements}
\end{table}
	\end{enumerate}
		
		
	
	
	
	
	
	
	

	
	
	


 

