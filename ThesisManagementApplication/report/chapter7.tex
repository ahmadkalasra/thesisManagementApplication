\chapter{Conclusions}\label{chap:conclusions}
%\version{v1.10.2015}

\section*{}
Reading the information directly from XML file is not humanly possible. It is not designed for human readable and have strict syntax. For read the XML is necessary to parse XML file. The parsing is performed using different programming tools. 
\par Now a days organizations archive there inactive records in ZIP format which is basically generated XML files. If they want to change or check that inactive data then it is difficult to read XML directly. 

\par Loading the information back to a DBMS is required. This is expensive and require resources. For this purpose the technical user is required who have knowledge of databases and XML as well. Now again need to connect with DBMS and records load back to DBMS and user view the records. 

\par The development of tool ADD make it easy to read the XML without the requirement of technical knowledge of DBMS. Using ADD user not need to connected with database and load records again and again. It is very useful tool. ADD allows to look complete database without connecting to DBMS. ADD software gives accurate and useful data of the opened archived database. By using this application user not need to learn specific language of DBMS.

In chapter 1, described key factors (load XML file, parsing and joining algorithms). To implement these factors used Visual Studio IDE and use different API and display there output which are mentioned in chapter 5.
In this project, read compressed XML format database by using parsing and show data in tabular format. Parsing is elegant technique to read XML files. 

\par After completion of this application we are master in C\# programming language. We learned different API of C\#, process of XML which includes select the XML file, specific extensions of XML file, parsing XML and load the XML. learn different join algorithms, filtering and sorting procedures.
  


